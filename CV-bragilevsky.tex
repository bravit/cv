\documentclass[11pt,a4paper]{moderncv}

% moderncv themes
\moderncvtheme[grey]{classic}
\usepackage[utf8]{inputenc}                   % replace by the encoding you are using

% adjust the page margins
\usepackage[scale=0.88]{geometry}
%\setlength{\hintscolumnwidth}{3cm}						% if you want to change the width of the column with the dates
%\AtBeginDocument{\setlength{\maketitlenamewidth}{6cm}}  % only for the classic theme, if you want to change the width of your name placeholder (to leave more space for your address details
\AtBeginDocument{\recomputelengths}                     % required when changes are made to page layout lengths

% personal data
\firstname{Vitaly}
\familyname{Bragilevsky}
\title{Curriculum Vitae}
\email{mail@bravit.guru}
\homepage{bravit.guru}



%\nopagenumbers{}                             % uncomment to suppress automatic page numbering for CVs longer than one page


%----------------------------------------------------------------------------------
%            content
%----------------------------------------------------------------------------------
\begin{document}
\maketitle
\vspace{-10mm}


\section{Professional Experience}
\subsection{Academy}
\cventry{2019--2022}{Lecturer, Curriculum Designer}{Saint Petersburg State University, Department of Mathematics and Computer Science}{Saint Petersburg}{Russia}
{Courses: Programming Foundations (Julia, Kotlin), Operating Systems, Mathematical Logic in CS.}
\cventry{2021}{Lecturer}{Computer Science Center}{Saint Petersburg}{Russia}
{Courses: Mathematical Logic in CS, Practical Minimum for Software Developers.}
\cventry{2021}{Lecturer}{European University}{Saint Petersburg}{Russia}
{Courses: Data Visualization in Humanities (with R programming language).}
\cventry{2003--2019}{Lecturer, Curriculum Designer (since 2007)}{Southern Federal University (SFedU), I.I. Vorovich Institute of Mathematics, Mechanics and Computer Science}{Rostov-on-Don}{Russia}
{Courses: Functional Programming in Haskell (since 2009), Theory of Computations (since 2012), Theory of Programming Languages, Data Visualization, Abstract Algebra, Category Theory, Web/XML-technologies, Computer Networks, Operating Systems, Programming in (Java, Python), Computer Methods in Discrete Mathematics, \emph{etc}.}
\cventry{2015--2019}{Coordinator}{AWS Educate Initiative, iOS Developer University Program  (for SFedU)}{}{}{}
\cventry{2002--2017}{Teacher}{Computer School for children (Southern Federal University)}{Rostov-on-Don}{Russia}{}

\subsection{Software Development}
\cventry{2019--2022}{Software Developer, Team Lead}{JetBrains}{}{Saint Petersburg, Russia}{}
\cventry{2012--2013}{Software Developer}{Implementing vector operations within LLVM infrastructure}{Angstrem-SFedU laboratory}{}{}
\cventry{2012}{GHC Contributor}{Implementing \texttt{-interactive-print} feature in GHCi (Haskell)}{}{}{}

\section{Writing and Publishing}
\cventry{}{Author}{Haskell in Depth, Manning Publications}{2021}{}{}
\cventry{}{Technical Proofreader, External Reviewer}{}{Manning Publications}{}{}{}
\cventry{}{Translator, Editor}{}{DMK Press}{}{}{}
\subsection{Translations from English to Russian}
\cvlistitem{M. Lipovača. Learn you a Haskell for Great Good (\emph{editor}), 2012.}
\cvlistitem{G. Dowek, J.-J. Lévy. Introduction to the Theory of  Programming Languages (\emph{translator, with Artem Pelenitsyn)}, 2013.}
\cvlistitem{R. Bird. Pearls of Functional Algorithm Design (\emph{translator, with Artem Pelenitsyn)}, 2013.}
\cvlistitem{S. Marlow. Parallel and Concurrent Programming in Haskell (\emph{translator}), 2014.}
\cvlistitem{C. Okasaki. Purely Functional Data Structures (\emph{editor}), 2016.}
\cvlistitem{W. Kurt. Get Programming with Haskell (\emph{editor}), 2018.}

\section{Other activities}

\subsection{Guest Lecturing}
\cvlistitem{Programming in Idris with Dependent Types, University of Twente (Enschede, The Netherlands), May~2016; CS Club (St.~Petersburg, Russia), February~2017; Higher School of Economics (Moscow, Russia), November 2017.}
\cvlistitem{The Curry--Howard Correspondence: from Logic to Programming and Back Again, Summer School on Contemporary Mathematics (Dubna, Russia), July 2017.}
\cvlistitem{The Glasgow Haskell Compiler: Theory of Programming Languages at Work, CS Club (St.~Petersburg, Russia), March 2018.}
\cvlistitem{The Type Theory Behind the Glasgow Haskell Compiler Internals, LambdaConf (Boulder, CO, USA), June 2018.}
\cvlistitem{Type Inference from Hindley--Milner to GHC, CS Club (St.~Petersburg, Russia), March~2019.}
\cvlistitem{Introduction to the Theory of Programming Languages, CS Summer School (Novosibirsk, Academgorodok, Russia), June 2019.}

\subsection{Selected Conference Talks}

\cvlistitem{FPConf-2016 (Moscow, Russia), Haskell 2020: problems and perspectives.}
\cvlistitem{F(by)-2019 (Minsk, Belarus), Haskell and Type Theory: Better Together.}
\cvlistitem{BobKonf-2019 (Berlin, Germany), A Tutorial on Type-level programming in Haskell.}
\cvlistitem{AppsConf-2019 (Moscow, Russia), Don't bother me with functional programming.}
\cvlistitem{FPure-2019 (Kazan, Russia), Crash Course on Compiler Development in Haskell.}
\cvlistitem{\emph{Meetup and seminar talks} in Russia (Rostov-on-Don, St.~Petersburg) and USA~(Eugene, OR; Portland, OR; Boston, MA; New York, NJ).}

\subsection{Community Services}
\cventry{2018--2022}{GHC (Glasgow Haskell Compiler) Steering Committee Member}{}{}{}{}


\section{Academy Awards}
\cventry{Aug~2018-- Jan~2019}{Fulbright Faculty Development Program Grantee, Courtesy Research Assistant}{University of Oregon}{Eugene, OR}{USA}{}
\cventry{June~2017-- June~2018}{Vladimir Potanin Foundation Grantee}{Developing Graduate Course in Data Vi\-sua\-li\-zation}{Southern Federal University}{Rostov-on-Don, Russia}{}

\section{Education}
\cventry{2001--2003}{M.Sc. in Applied Mathematics and Computer Science}{Rostov State University}{Rostov-on-Don, Russia}
{\textit{with honours}, Singular-type operators in the spaces of vector-valued functions, supervised by prof. V.S. Pilidi}{}
\cventry{1997--2001}{B.\,Sc. in Applied Mathematics and Computer Science}{Rostov State University}{Rostov-on-Don, Russia}
{\textit{with honours}}{}
\subsection{Other}
%\cvlistitem{Short course in methods, tools and systems for information security,   Moscow Engineering Physics Institute, Moscow, Russia, December 2004.}
%\cvlistitem{Short course in hardware and software for computer networking, 
%Southern Federal University, Rostov-on-Don, Russia, November--December 2010.}
\cvlistitem{DeepSpec Summer School on Verified Systems, Princeton, NJ, USA, July~16--July~27, 2018.}
\cvlistitem{Summer School Marktoberdorf 2011 
(Tools for Analysis and Verification of Software Safety and Security), 
Bayrischzell, Germany, August 2--14, 2011.}
\cvlistitem{10th Annual Oregon Programming Languages Summer School
(Types, Semantics and Verification), University of Oregon, Eugene, OR, USA, June~16--July~1, 2011.}
%\cvlistitem{Short course in iOS-applications Development, National Research University «MPEI», Moscow, Russia, June 14--19, 2015.}
%\cvlistitem{Online courses (coursera.org, edx.org) in Automata, Logic, Complilers, Machine Learning, Algorithms, Cryptography, Functional Programming, etc.)}


% \section{Master thesis}
% \cvline{title}{\emph{Singular-type operators in the spaces of vector-valued functions.}}
% \cvline{supervisor}{Prof. V.S. Pilidi}
% \cvline{description}{\small Algebras of operators of singular and bisingular type are considered. For these algebras, 
% the notions of an envelope of a family of operators are introduced, 
% necessary and sufficient conditions of its existence and uniqueness are given. 
% There are obtained estimates for the essential norm of the envelope operator. 
% The statements of the paper are demonstrated for the case of singular integral 
% operators with continuous operator-valued coefficients.}


% \section{Selected Papers}
% \cvlistitem{V. Bragilevsky, V. S. Pilidi. Analog of Simonenko's theorem on an envelope of a family of operators for operators on the spaces of vector-valued functions.
% Izvestia vuzov. Severokavkazsky region. Vol. 4. 2005. (\emph{in Russian})}
% \cvlistitem{V. Bragilevsky. Limits of Folds Expressiveness. Practice of Functional Programming. Vol.~4. 2010. (\emph{in Russian})}

% \section{Interests}
% \cvline{History}{\small Medieval Europe}
% \cvline{Literature}{\small Contemporary writers}
% \cvline{Music}{\small 19th century music, opera}

\end{document}

%%% Local Variables:
%%% mode: latex
%%% TeX-master: t
%%% End:
